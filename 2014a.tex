\documentclass[fleqn,twocolumn,9pt]{jsarticle}

\usepackage{csatmath}
\usepackage{newtxtext}
\usepackage{newtxmath}
\usepackage[multi]{otf}

\begin{document}

%%----------------

\Question  % 1
題意より,$2+a+1+2-3=6$.\quad $\therefore\ a = \Ans{4}$.

%%----------------

\Question  % 2
$\cos$と$\tan$の関係より
\begin{equation*}
  \frac{1}{\cos^2 \theta} = \tan^2 \theta + 1 = \left(\frac{\sqrt{5}}{5}\right)^2 + 1 = \frac{6}{5}
\end{equation*}
倍角の公式より
\begin{equation*}
  \cos 2\theta = 2 \cos^2 \theta - 1 = 2 \cdot \frac{5}{6} - 1 = \Ans{\frac{2}{3}}
\end{equation*}

%%----------------

\Question  % 3
$x$座標,$y$座標にそれぞれ着目して,
\begin{gather*}
  \frac{2}{5}a + \frac{3}{5} \cdot (-2) = 0 \Therefore{a = 3} \\
  \frac{2}{5} \cdot 5 + \frac{3}{5} \cdot 0 = b \Therefore{b = 2}
\end{gather*}
したがって,$a+b=\Ans{5}$.

%%----------------

\Question  % 4
公差を$d$とおくと,$a_9=3a_3$より
\begin{equation*}
  2+8d = 3(2+2d) \Therefore{d = 2}
\end{equation*}
したがって,$a_5=2+4d=\Ans{10}$.

%%----------------

\Question  % 5
$A^C \cup B^C$の余事象は$A \cap B$なので,
\begin{align*}
  P(A) &= P(A \cap B) + P(A \cap B^C) \\
  &= \left(1 - \frac{4}{5}\right) + \frac{1}{4} = \frac{9}{20}
\end{align*}
したがって,
\begin{equation*}
  P(A^C) = 1 - P(A) = 1 - \frac{9}{20} = \Ans{\frac{11}{20}}
\end{equation*}

%%----------------

\Question  % 6
直線$x=4-y=z-1$上に2点$\Pt{P}(0,4,1)$,$\Pt{Q}(1,3,2)$をとると,$\Vect{BA}=(5,5,a-3)$と$\Vect{QP}=(1,-1,1)$が直交することから
\begin{equation*}
  \Vect{BA} \cdot \Vect{QP} = 5 \cdot 1 + 5 \cdot (-1) + (a-3) \cdot 1 = 0
\end{equation*}
これより$a-3=0$,すなわち$a=\Ans{3}$を得る.

%%----------------

\Question  % 7
$f(x)$を変形すると,
\begin{align*}
  f(x) &= 2 \cos^2 x + k \sin 2x - 1 \\
  &= \cos 2x + k \sin 2x \\
  &= \sqrt{1+k^2} \sin (2x + \alpha)
\end{align*}
ただし,$\alpha$は$\sin\alpha = 1 / \sqrt{1+k^2}$を満たす角.$\sin(2x+\alpha)$の最大値は1であるから,題意より
\begin{equation*}
  \sqrt{1+k^2} = \sqrt{10} \Therefore{k = \Ans{3}}
\end{equation*}

%%----------------

\Question  % 8
与えられた方程式より$(m_1,m_2)=(1/2,1)$.また,$y^2=8x \EqLabel{08-y}$の両辺を$x$で微分すると,
\begin{equation*}
  2y \cdot \frac{dy}{dx} = 8 \EqLabel{08-y'}
\end{equation*}
\EqRef{08-y},\EqRef{08-y'}より,傾きが$m_1=1/2$である接線の接点は$(8,8)$であるとわかるので,この接線の方程式は,
\begin{equation*}
  y = \frac{1}{2}(x-8) + 8 = \frac{1}{2}x + 4 \EqLabel{08-L1}
\end{equation*}
同様にして,傾きが$m_2=1$である接線の方程式は,
\begin{equation*}
  y = (x-2) + 4 = x + 2 \EqLabel{08-L2}
\end{equation*}
\EqRef{08-L1},\EqRef{08-L2}より,交点の$x$座標は$\Ans{4}$である.

%%----------------

\Question  % 9
4が0個のときは,1,2,3から5個の数字を選ぶことになるので,場合の数は
\begin{equation*}
  \nHr{3}{5} = \nCr{7}{2} = 21
\end{equation*}
4が1個のときは,1,2,3から4個の数字を選ぶことになるので,場合の数は
\begin{equation*}
  \nHr{3}{4} = \nCr{6}{2} = 15
\end{equation*}
したがって,求める場合の数は$21 + 15 = \Ans{36}$.

%%----------------

\Question  % 10
与えられた不等式を満たすことは,
\begin{itemize}
  \item $f(x) > 0$かつ$f(x) \ge g(x)$
  \item $f(x) < 0$かつ$f(x) \le g(x)$
\end{itemize}
のいずれかを満たすことと同値である.図より,条件を満たす整数$x$は$-4$,$-1$,$1$,$3$,$4$の$\Ans{5}$個である.

%%----------------

\Question  % 11
題意より,
\begin{equation*}
  f(n) = \frac{1}{n(n+1)}
\end{equation*}
また,
\begin{align*}
  g(n) &= b_1 + \sum_{k=1}^{n-1} \frac{1}{k(k+1)} \\
  &= 1 + \sum_{k=1}^{n-1} \left(\frac{1}{k} - \frac{1}{k+1}\right) \\
  &= 1 + \left(1 - \frac{1}{2}\right) + \left(\frac{1}{2} - \frac{1}{3}\right) + \cdots + \left(\frac{1}{n-1} - \frac{1}{n}\right) \\
  &= 2 - \frac{1}{n}
\end{align*}
したがって,
\begin{equation*}
  f(4) = \frac{1}{20}, \quad g(10) = \frac{19}{10}, \quad \frac{g(10)}{f(4)} = \Ans{38}
\end{equation*}

%%----------------

\Question  % 12
$h(x)=f(x)g(x)$とおく.$h(x)$が$x=0$において連続であることより,
\begin{equation*}
  \lim_{x \to 0} h(x) = \lim_{x \to 0} \frac{f(x)}{\ln (x+1)}
\end{equation*}
は存在する.よって$f(0)=0$であり,$f(x)$の条件から$f(x)=x^2+ax$とおける.ロピタルの定理より,
\begin{equation*}
  \lim_{x \to 0} h(x) = \lim_{x \to 0} \frac{x^2+ax}{\ln (x+1)} = \lim_{x \to 0} \frac{2x+a}{1/(x+1)} = a
\end{equation*}
一方,
\begin{equation*}
  h(0) = f(0) \cdot g(0) = 0 \cdot 8 = 0
\end{equation*}
であるので,再び$h(x)$の$x=0$における連続性により$a=0$.ゆえに$f(x)=x^2$であり,$f(3)=\Ans{9}$.

%%----------------

\Question  % 13
$l$の方程式は$x^2=(y+1)^2$,$C$の方程式は$x^2=2y-1$と変形できる.交点は$(1,0)$と$(3,2)$であるから,
\begin{align*}
  V &= \int_0^2 \{ \pi (y+1)^2 - \pi (2y^2-1) \} \, dy \\
  &= \pi \int_0^2 (2y - y^2) \, dy \\
  &= \pi \left[ y^2 - \frac{1}{3}y^3 \right]_0^2 = \Ans{\frac{4}{3}\pi}
\end{align*}

%%----------------

\Question  % 14
$C$の方程式を標準形に改めると
\begin{equation*}
  \frac{x^2}{1} + \frac{y^2}{(1/\sqrt{2})^2} = 1
\end{equation*}
となるので,$c < 0$より
\begin{equation*}
  c = -\sqrt{1^2 + \left(\frac{1}{\sqrt{2}}\right)^2} = -\sqrt{\frac{3}{2}}
\end{equation*}
一方,題意より,$\Pt{F}(c,0)$を原点中心に$-\theta$だけ回転させた点$(c \cos\theta, -c \sin\theta)$は直線$l$上にあるので,
\begin{equation*}
  c \cos\theta - (- c \sin\theta) - 1 = 0
\end{equation*}
すなわち
\begin{equation*}
  \cos\theta + \sin\theta = \frac{1}{c}
\end{equation*}
ここで
\begin{equation*}
  (\cos\theta + \sin\theta)^2 = 1 + 2 \cos\theta \sin\theta = 1 + \sin 2\theta
\end{equation*}
であるから,
\begin{equation*}
  \sin 2\theta = \left(\frac{1}{c}\right)^2 - 1 = \Ans{-\frac{1}{3}}
\end{equation*}

%%----------------

\Question  % 15
辺$\Pt{A_2D_2}$を延長して辺$\Pt{C_1D_1}$と交差する点を$\Pt{E_1}$とする.$\Line{A_2B_2}=x$とおくと,図のように$\triangle\Pt{D_1D_2E_1}$の各辺の長さが決まる.
\Fig{2014a-fig.15}
ピタゴラスの定理より,
\begin{equation*}
  (2-3x)^2 + (1-x)^2 = 1^2
\end{equation*}
整理すると,
\begin{equation*}
  10x^2 - 14x + 4 = 2(5x-2)(x-1) = 0
\end{equation*}
$0 < x < 1$より$x=2/5$である.これより,$R_{n+1}$で新たに色が塗られる部分の面積は,$R_n$のときの$4/25$倍であるとわかるので($n \ge 1$),
\begin{align*}
  S_\infty &= S_1 + \frac{4}{25} S_1 + \left(\frac{4}{25}\right)^2 S_1 + \cdots \\
  &= \frac{1}{1-(4/25)} S_1 \\
  &= \Ans{\frac{25}{21}\left(\frac{\pi}{2}-1\right)}
\end{align*}

%%----------------

\Question  % 16
確率の公理より,
\begin{equation*}
  P(0 \le X \le a) = ka^2 = 1 \EqLabel{16-P(X)}
\end{equation*}
また,確率密度関数を$f(x)$で表すと,
\begin{equation*}
  f(x) = \bigl( kx^2 \bigr)' = 2kx
\end{equation*}
であるので,
\begin{align*}
  E(X) &= \int_0^a x \cdot f(x) \, dx \\
  &= \int_0^a 2kx^2 \,dx \\
  &= \frac{2}{3}ka^3 = 1 \EqLabel{16-E(X)}
\end{align*}
\EqRef{16-P(X)},\EqRef{16-E(X)}より,$a=3/2$,$k=\Ans{4/9}$.

%%----------------

\Question  % 17
$AB+A^2B = E$より,
\begin{equation*}
  (A+A^2)B = E \Therefore{B^{-1} = A+A^2}
\end{equation*}
よって,\CIDK{358}は真である.また,同様に$A^{-1}$も存在するので,
\begin{gather*}
  (AB)^{-1} = B^{-1}A^{-1} = (A+A^2)A^{-1} = E+A \\
  (BA)^{-1} = A^{-1}B^{-1} = A^{-1}(A+A^2) = E+A
\end{gather*}
したがって,$AB = BA = (E+A)^{-1}$であるから,\CIDK{361}も真である.さらに,$(A-E)^2+B^2=O$より,
\begin{equation*}
  (A+A^2)^2 (A-E)^2 + (A+A^2)^2 B^2 = O
\end{equation*}
ここで,$(A+A^2) B = E$より,
\begin{equation*}
  \text{(左辺)} = A^6 - 4A^4 + A^2 + E = (A^3-A)^2 + E
\end{equation*}
したがって,\CIDK{364}も真である.すなわち,\CIDK{358},\CIDK{361},\CIDK{364}はすべて真である.

%%----------------

\Question  % 18
開区間$(k\pi - \pi/2, k\pi + \pi/2)$($k$は整数)において,$y = \tan x$は実数全体を値域とする単調増加な関数であるので,$y = n$となる$x$の値が区間内にただひとつ存在する.したがって,
\begin{equation*}
  n\pi - \frac{3}{2}\pi \le a_n \le n\pi - \frac{1}{2}\pi
\end{equation*}
ここで,
\begin{equation*}
  \lim_{n \to \infty} \frac{1}{n}\left( n\pi - \frac{3}{2}\pi \right) = \lim_{n \to \infty} \frac{1}{n}\left( n\pi - \frac{1}{2}\pi \right) = \pi
\end{equation*}
であるから,はさみうちの原理により,
\begin{equation*}
  \lim_{n \to \infty} \frac{a_n}{n} = \Ans{\pi}
\end{equation*}

%%----------------

\Question   % 19
$S$を$xy$平面で切断したとき,断面の円の面積は$64\pi$であるから半径は$8$であり,$x$軸,$y$軸の両方に接することから中心は$(8,8)$に存在する.よって,$S$の中心の座標は$(8,8,z)$とおける.

$S$を$xz$平面で切断すると,断面は図のようになる.
\Fig{2014a-fig.19}
$S$と$z$軸との2つの交点を$\Pt{Z_1}$,$\Pt{Z_2}$,中点を$\Pt{M}$,円の中心を$\Pt{C}$とおくと,$\triangle\Pt{Z_1CM}$は直角三角形となり,
\begin{equation*}
  \Line{Z_1C} = z, \hquad \Line{CM} = 8, \hquad \Line{MZ_1} = 4 \hquad (\because\ \Line{Z_1Z_2} = 8)
\end{equation*}
であるから,
\begin{equation*}
  z^2 = 8^2 + 4^2 = 80 \Therefore{z = 4\sqrt{5}}
\end{equation*}
すなわち,$S$の中心の座標は$(8,8,4\sqrt{5})$である.

ここで,求める半径を$r$とおくと,$(8,0,0)$が$S$の表面上の一点であることから,
\begin{equation*}
  r^2 = 8^2 + \left( 4\sqrt{5} \right)^2 = 144 \Therefore{r=\Ans{12}}
\end{equation*}

%%----------------

\Question  % 20
$h(x)=3f(x)+5g(x)$とおく.題意より,$f(x)$は整数であり,また$0 \le g(x) < 1$であるので,
\begin{itemize}
  \item $h(x)=10 \Rightarrow (f(x),g(x))=(2,0.8),\,(3,0.2)$
  \item $h(x)=20 \Rightarrow (f(x),g(x))=(6,0.4)$
  \item $h(x)=30 \Rightarrow (f(x),g(x))=(9,0.6),\,(10,0)$
  \item $h(x)=40 \Rightarrow (f(x),g(x))=(12,0.8),\,(13,0.2)$
\end{itemize}
よって,$\log a = 3.2$,$\log b = 12.8$であるので,
\begin{equation*}
  \log ab = \log a + \log b = 3.2 + 12.8 = \Ans{16}
\end{equation*}

%%----------------

\Question  % 21
題意より,
\begin{equation*}
  f'(x) = \frac{\pi}{2} f(x+1)
\end{equation*}
よって,求める積分の値を$I$とすると,
\begin{align*}
  I &= \pi^2 \int_0^1 x f(x+1) \, dx \\
  &= 2\pi \int_0^1 x f'(x) \, dx \\
  &= 2\pi \Bigl[ xf(x) \Bigr]_0^1 - 2\pi \int_0^1 x'f(x) \, dx \\
  &= 2\pi - 2\pi \int_0^1 f(x) \, dx
\end{align*}
ここで,$f(x)$は原点対称より$f(-1)=-1$であり,
\begin{align*}
  2\pi \int_0^1 f(x) \, dx &= -4 \cdot \frac{\pi}{2} \int_1^{-1+1} f(x) \, dx \\
  &= -4f(-1) = 4
\end{align*}
したがって,$I=2\pi-4=\Ans{2(\pi-2)}$.

%%----------------

\Question  % 22
合成関数の微分により,
\begin{equation*}
  f'(x)=5 \cdot (3x-3)' \cdot e^{3x-3} = 15e^{3x-3}
\end{equation*}
したがって,$f'(1) = \Ans{15}$.

%%----------------

\Question  % 23
女性は全部で15人であり,そのうち完走者は9人なので,
\begin{equation*}
  p = \frac{9}{15} = \frac{3}{5} \Therefore{100p = \Ans{60}}
\end{equation*}

%%----------------

\Question  % 24
$x(x-3)=X$とおくと,与えられた方程式は
\begin{equation*}
  \sqrt{2X} = X-4 \EqLabel{24-eq}
\end{equation*}
と書ける.両辺を二乗して整理すると,
\begin{equation*}
  X^2-10X+16 = (X-2)(X-8) = 0
\end{equation*}
\EqRef{24-eq}より$X-4 \ge 0$なので$X=8$.よって,$k$は
\begin{equation*}
  X-8 = x^2-3x-8 = 0
\end{equation*}
の2つの解の積であり,解と係数の関係より$k=-8$であるから,$k^2=\Ans{64}$.

%%----------------

\Question  % 25
題意より,
\begin{equation*}
  1 - k \log \frac{R^{27/23}}{R} = \frac{v_c}{v_c/2}
\end{equation*}
すなわち
\begin{equation*}
  1 - \frac{4}{23} k \log R = 2 \EqLabel{24-2x}
\end{equation*}
同様にして,
\begin{equation*}
  1 - (a-1) k \log R = 3 \EqLabel{24-3x}
\end{equation*}
\EqRef{24-2x},\EqRef{24-3x}より,
\begin{equation*}
  -k \log R = \frac{23}{4} = \frac{2}{a-1} \Therefore{23a = \Ans{31}}
\end{equation*}

%%----------------

\Question  % 26
$n$が十分に大きいと仮定して,二項分布を
\begin{equation*}
  \mu = p = 0.8, \quad \sigma = \sqrt{\frac{p(1-p)}{n}} = \frac{0.4}{\sqrt{n}}
\end{equation*}
の正規分布で近似すると,信頼区間に関する条件から
\begin{equation*}
  1.96 \cdot \frac{0.4}{\sqrt{n}} = \frac{0.098}{2} \Therefore{n = \Ans{256}}
\end{equation*}

%%----------------

\Question  % 27
楕円の性質より,$\Line{FP}+\Line{F'P}=10$であるから,
\begin{equation*}
  \Line{AP} - \Line{FP} = \Line{AP} - (10 - \Line{F'P}) = \Line{AP} + \Line{F'P} - 10
\end{equation*}
ここで$\Line{AP} + \Line{F'P}$が最小になるのは$\Pt{A}$,$\Pt{P}$,$\Pt{F'}$が一直線上に並ぶときであり,$\Line{AP} - \Line{FP}$の最小値が1であることより$\Line{AF'}=11$である.したがって,
\begin{equation*}
  a^2 + 4^2 = 11^2 \Therefore{a^2 = \Ans{105}}
\end{equation*}

%%----------------

\Question  % 28
$\triangle\Pt{ABC}$が二等辺三角形であるので
\begin{equation*}
  \Line{AC} = \frac{2}{\sin(\theta/2)}
\end{equation*}
$\Pt{P}$から$\Pt{AD}$に引いた垂線の足を$\Pt{H}$とすると,
\begin{align*}
  S(\theta) &= \frac{1}{2} \cdot \Line{BD} \cdot \Line{PH} \\
  &= \frac{1}{2} (\Line{AD} - 4) (\Line{AP} \sin 2\theta) \\
  &= \frac{1}{2} \left( \frac{2}{\sin(\theta/2)} - 4 \right) \cdot \frac{2}{\sin(\theta/2)} \cdot \sin 2\theta \\
  &= 16 \cos\frac{\theta}{2} \cos\theta \left( \frac{1/2}{\sin(\theta/2)} - 1 \right)
\end{align*}
したがって,
\begin{equation*}
  \lim_{\theta \to 0} \theta S(\theta) = \Ans{16}
\end{equation*}

%%----------------

\Question  % 29
平面$y=4$,$y+3\sqrt{z}+8=0$に射影された点の$x$座標はもとの点の$x$座標に等しいから,$\Vect{PQ}$,$\Vect{P_1Q_1}$,$\Vect{P_2Q_2}$の$x$成分はみな等しい\EqLabel{29-1}.そこで,
\begin{equation*}
  \Vect{PQ}=(x, r \cos\theta, r \sin\theta) \quad \bigl( r \ge 0, \hquad 0 \le \theta < 2\pi \bigr)
\end{equation*}
とおく.与えられた座標空間を$yz$平面に射影すると以下の図のようになる.
\Fig{2014a-fig.29}
\EqRef{29-1}および上図より,
\begin{align*}
  \text{(与式)} &= 2|\Vect{PQ}|^2 - |\Vect{P_1Q_1}|^2 - |\Vect{P_2Q_2}|^2 \\
  &= 2(x^2 + r^2) - (x^2 + r^2 \sin^2 \theta) \\
  & \qquad - \left\{ x^2 + r^2 \cos^2 \left(\theta + \frac{\pi}{6}\right) \right\} \\
  &= r^2 \left\{ 2 - \sin^2 \theta - \cos^2 \left(\theta + \frac{\pi}{6}\right) \right\} \\
  &= r^2 \left\{ 1 + \frac{1}{2} \cos 2\theta - \frac{1}{2} \cos \left(2\theta + \frac{\pi}{3}\right) \right\} \\
  &= r^2 \left\{ 1 + \sin \frac{\pi}{6} \sin \left(2\theta + \frac{\pi}{6}\right) \right\}
\end{align*}
題意より$r \le |\Vect{PQ}| \le 4$なので,与式は$r=4$,$\theta=\pi/6$のときに最大値$\Ans{24}$をとる.

%%----------------

\Question  % 30
$g(x)=f(x)e^{-x}$より,
\begin{gather*}
  g'(x) = \{ f'(x) - f(x) \} e^{-x} \\
  g''(x) = \{ f''(x) - 2f'(x) + f(x) \} e^{-x}
\end{gather*}
ここで,$f(x)=ax^2+bx+c$とおくと,(\CIDK{1086})より
\begin{gather*}
  g''(1) = (-a-b+c) e^{-1} = 0 \\
  g''(4) = (2a+2b+c) e^{-4} = 0
\end{gather*}
これより$b=-a$,$c=0$,すなわち,
\begin{equation*}
  g(x) = ax(x-1)e^{-x}
\end{equation*}

いま,点$(0,k)$から$y = g(x)$に引いた接線の接点の座標を$(t,g(t))$とおくと,接線の方程式は$y = g'(t) x + k$となり,これが点$(t,g(t))$を通ることから
\begin{equation*}
  k = g(t) - g'(t) \cdot t = a (t^3-2t^2) e^{-t}
\end{equation*}
が成り立つ.そこで右辺を$h(t)$とおくと,(\CIDK{1377})より方程式$h(t) = k$は$-1 < k < 0$において相異なる3つの実数解を持つ\EqLabel{30-cond}.ここで,
\begin{align*}
  h'(t) &= a (3t^2-4t) e^{-t} - a (t^3-2t^2) e^{-t} \\
  &= -a t (t-1) (t-4) e^{-t}
\end{align*}
であることに注意すると,

\Case{a}%
$a > 0$のとき,$t > 2$ならば$h(t) > 0$なので,\EqRef{30-cond}における実数解は$t \le 2$の範囲に存在する.この範囲における$h(t)$の増減は以下のとおりである.
\begin{equation*}
  \def\C#1{\makebox[1.2em][c]{$#1$}}%
  \begin{array}{|r||c|c|c|c|c|c|c|}
    \hline
    t\  & \C{-\infty} & \C{\cdots} & \C{0} & \C{\cdots} & \C{1} & \C{\cdots} & \C{2} \\
    \hline
    h'(t) & & + & 0 & - & 0 & \multicolumn{2}{c|}{+} \\
    \hline
    h(t) & -\infty & \nearrow & 0 & \searrow & h(1) & \nearrow & 0 \\
    \hline
  \end{array}
\end{equation*}
\EqRef{30-cond}より,$h(1)=-ae^{-1}=-1$,すなわち$a=e$.

\Case{b}%
$a < 0$のとき,同様にして考えると,\EqRef{30-cond}を満たすような$a$は存在しないことがわかる.

以上より,
\begin{equation*}
  g(x) = ex(x-1) e^{-x} = x(x-1)e^{-x+1}
\end{equation*}
であるので,
\begin{equation*}
  g(-2) \cdot g(4) = 6e^3 \cdot 12e^{-3} = \Ans{72}
\end{equation*}

%%----------------

\end{document}
